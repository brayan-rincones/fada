\documentclass{article}
\usepackage[utf8]{inputenc}
\usepackage[letterpaper, margin=1in]{geometry}

\title{Miniproyecto FADA - Universidad del Valle}
\date{}
\author{David Santiago Cortés, Alejandro Orozco, Brayan Rincones}

\begin{document}
	\maketitle

	\section{Soluciones Planteadas}
		\textbf{Idea General de la solución:}\\
		Almacenar los animales y sus grandezas en una estructura de datos tipo lista, arreglo o diccionario, 
		se ordena esta estructura de acuerdo a los valores de las grandezas y se guarda utiliza para armar las
		escenas de todas las partes del evento.\\	
		Para calcular la apertura se realiza un primer ciclo for que se ejecutará $(m-1)*k$ veces, iniciando en $i=1$, en cada ciclo se calcula
		una escena para la apertura así:
		\begin{center}
			\texttt{escena = [ animales[i], animales[i+1], animales[i+2] ]} \\
			ó\\	
			\texttt{escena = [ animales[n], animales[n-1], animales[n-2] ]}\\
			Dependiendo de sí \texttt{animales} se ordeno ascendentemente o descendentemente.
		\end{center}
		Luego, se añade la escena recién calculada al arreglo \texttt{apertura}, que va a contener todas las escenas de la apertura
		y que por consiguiente también se usarán en las partes posteriores. Luego de que \texttt{apertura} este lleno (o mientras se esté llenando)
		se deben buscar posibles empates y resolverlos. Para calcular las demás partes se sigue un proceso similar.\\
		La forma en que se van a calcular los datos que requiere el gerente del zoológico son particulares de cada solución.

		
		\subsection{$O(n^2)$}
			\textbf{Idea de la solución:}\\
			\textbf{Estructuras de datos utilizadas:} \\
			\textbf{Lenguaje en el que se implementó:}
		
		\subsection{$O(n*\log(n))$}
			\textbf{Idea de la solución:}\\
			\textbf{Estructuras de datos utilizadas:} \\
			\textbf{Lenguaje en el que se implementó:}
		\subsection{$O(n)$}
			\textbf{Idea de la solución:} Después de recolectar los datos de entrada se construye una lista de tuplas a partir de la
			lista de animales y grandezas, luego se ordena ascendentemente de acuerdo a los valores de cada tupla utilizando Counting Sort.
			Para realizar el ordenamiento de escenas localmente, escenas y partes del evento se va a usar una aproximación \textit{top-down}
			de la siguiente manera.
				\begin{enumerate}
					\item Calcular la grandeza de cada parte y almacenar estas grandezas en una lista auxiliar.
					\item Armar una lista de tuplas, cada tupla de la forma: (parte:grandeza) y ordenar la lista con Counting Sort.
					\item Para cada parte, armar otra lista de tuplas (escena:grandeza) y ordenar la lista con Counting Sort.
					\item Realizar un procedimiento similar para las escenas de cada parte.
				\end{enumerate}
			\textbf{Estructuras de datos utilizadas:} Tuplas y listas.\\
			\textbf{Lenguaje en el que se implementó:} Python
	\section{Análisis de Resultados}
		\subsection{$O(n^2)$}
		\subsection{$O(n*\log(n))$}
		\subsection{$O(n)$}
	\section{Instrucciones para la ejecución}

	\section{Sets de prueba}

	\section{Conclusiones del proyecto}

\end{document}
